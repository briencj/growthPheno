% Options for packages loaded elsewhere
\PassOptionsToPackage{unicode}{hyperref}
\PassOptionsToPackage{hyphens}{url}
%
\documentclass[
]{article}
\usepackage{amsmath,amssymb}
\usepackage{lmodern}
\usepackage{iftex}
\ifPDFTeX
  \usepackage[T1]{fontenc}
  \usepackage[utf8]{inputenc}
  \usepackage{textcomp} % provide euro and other symbols
\else % if luatex or xetex
  \usepackage{unicode-math}
  \defaultfontfeatures{Scale=MatchLowercase}
  \defaultfontfeatures[\rmfamily]{Ligatures=TeX,Scale=1}
\fi
% Use upquote if available, for straight quotes in verbatim environments
\IfFileExists{upquote.sty}{\usepackage{upquote}}{}
\IfFileExists{microtype.sty}{% use microtype if available
  \usepackage[]{microtype}
  \UseMicrotypeSet[protrusion]{basicmath} % disable protrusion for tt fonts
}{}
\makeatletter
\@ifundefined{KOMAClassName}{% if non-KOMA class
  \IfFileExists{parskip.sty}{%
    \usepackage{parskip}
  }{% else
    \setlength{\parindent}{0pt}
    \setlength{\parskip}{6pt plus 2pt minus 1pt}}
}{% if KOMA class
  \KOMAoptions{parskip=half}}
\makeatother
\usepackage{xcolor}
\usepackage[margin=1in]{geometry}
\usepackage{graphicx}
\makeatletter
\def\maxwidth{\ifdim\Gin@nat@width>\linewidth\linewidth\else\Gin@nat@width\fi}
\def\maxheight{\ifdim\Gin@nat@height>\textheight\textheight\else\Gin@nat@height\fi}
\makeatother
% Scale images if necessary, so that they will not overflow the page
% margins by default, and it is still possible to overwrite the defaults
% using explicit options in \includegraphics[width, height, ...]{}
\setkeys{Gin}{width=\maxwidth,height=\maxheight,keepaspectratio}
% Set default figure placement to htbp
\makeatletter
\def\fps@figure{htbp}
\makeatother
\setlength{\emergencystretch}{3em} % prevent overfull lines
\providecommand{\tightlist}{%
  \setlength{\itemsep}{0pt}\setlength{\parskip}{0pt}}
\setcounter{secnumdepth}{-\maxdimen} % remove section numbering
\ifLuaTeX
  \usepackage{selnolig}  % disable illegal ligatures
\fi
\IfFileExists{bookmark.sty}{\usepackage{bookmark}}{\usepackage{hyperref}}
\IfFileExists{xurl.sty}{\usepackage{xurl}}{} % add URL line breaks if available
\urlstyle{same} % disable monospaced font for URLs
\hypersetup{
  pdftitle={growthPheno},
  hidelinks,
  pdfcreator={LaTeX via pandoc}}

\title{growthPheno}
\author{}
\date{\vspace{-2.5em}}

\begin{document}
\maketitle

\href{http://www.repostatus.org/\#active}{\includegraphics{http://www.repostatus.org/badges/latest/active.svg}}
\href{https://cran.r-project.org/}{\includegraphics{https://img.shields.io/badge/R\%3E\%3D-3.5.0-6666ff.svg}}
\href{https://cran.r-project.org/package=growthPheno}{\includegraphics{http://www.r-pkg.org/badges/version/growthPheno}}
\href{/commits/master}{\includegraphics{https://img.shields.io/badge/Package\%20version-2.1.11-orange.svg?style=flat-square}}
\href{/commits/master}{\includegraphics{https://img.shields.io/badge/last\%20change-2022--10--08-yellowgreen.svg}}
\href{http://choosealicense.com/licenses/gpl-2.0/}{\includegraphics{https://img.shields.io/badge/license-GPL\%20(\%3E\%3D2)-green.svg}}
\href{commits/master}{\includegraphics{https://cranlogs.r-pkg.org/badges/last-week/growthPheno}}

\texttt{growthPheno} is an R package that collects together functions
that can be used to perform functional analyses of phenotypic growth
data to smooth and extract traits, as described by Brien et al.~(2020).
Many of the functions can be applied to longitudinal data in general.

An overview can be obtained using \texttt{?growthPheno}. .

\hypertarget{more-information}{%
\subsection{More information}\label{more-information}}

Two vignettes, \texttt{Tomato} and \texttt{Rice}, illustrate the process
for smoothing and extraction of traits (SET), the former being the
example presented in Brien et al.~(2020). Use
\texttt{vignette("Tomato",\ package\ =\ "growthPheno")} or
\texttt{vignette("Rice",\ package\ =\ "growthPheno")} to access either
of the vignettes.

\hypertarget{installing-the-package}{%
\subsection{Installing the package}\label{installing-the-package}}

\hypertarget{from-a-repository-using-drat}{%
\subsubsection{\texorpdfstring{From a repository using
\texttt{drat}}{From a repository using drat}}\label{from-a-repository-using-drat}}

Windows binaries and source tarballs of the latest version of
\texttt{growthPheno} are available for installation from my
\href{http://chris.brien.name/rpackages}{repository}. Installation
instructions are available there.

\hypertarget{directly-from-github}{%
\subsubsection{Directly from GitHub}\label{directly-from-github}}

\texttt{growthPheno} is an R package available on GitHub, so it can be
installed from the RStudio console or an R command line session using
the \texttt{devtools} command \texttt{install\_github}. First, make sure
\texttt{devtools} is installed, which, if you do not have it, can be
done as follows:

\texttt{install.packages("devtools")}

Next, install \texttt{growthPheno} from GitHub by entering:

\texttt{devtools::install\_github("briencj/growthPheno")}.

The version of the package on CRAN (see CRAN badge above) and its
dependencies can be installed using:

\texttt{install.packages("growthPheno")}

If you have not previously installed \texttt{growthPheno} then you may
need to install it dependencies:

\texttt{install.packages(c("dae","GGally","ggplot2","grDevices","Hmisc","JOPS","methods","RColorBrewer","readxl","reshape","stringi"))}

\hypertarget{what-is-does}{%
\subsection{What is does}\label{what-is-does}}

This package can be used to perform a functional analysis of growth data
using splines to smooth the trend of individual plant traces over time
and then to extract traits for further analysis. This process is called
smoothing and extraction of traits (SET) by Brien et al.~(2020), who
detail the use of \texttt{growthPheno} for carrying out the method.
However, \texttt{growthPheno} now has the two wrapper, or primary,
functions \texttt{traitSmooth} and \texttt{traitExtractFeatures} that
implement the SET approach. These may be the only functions that are
used in that the complete SET process can be carried out using only
them. The \texttt{Tomato} vignette illustrates their use for the example
presented in Brien et al.~(2020).

The function \texttt{traitSmooth} utilizes the secondary functions
\texttt{probeSmooths}, \texttt{plotSmoothsComparison} and
\texttt{plotSmoothsMedianDevns} and accepts the arguments of the
secondary functions. The function \texttt{probeSmooths} utilizes the
tertiary functions \texttt{byIndv4Times\_SplinesGRs} and
\texttt{byIndv4Times\_GRsDiff}, which in turn call the function
\texttt{smoothSpline}. The function \texttt{plotSmoothsComparison} calls
\texttt{plotDeviationsBoxes}. All of these functions play a role in
choosing the smoothing method and parameters.

The primary function \texttt{traitExtractFeatures} uses the secondary
functions \texttt{getTimesSubset} and the set of \texttt{byIndv4Intvl\_}
functions. These functions are concerned with the extraction of traits
that have a single value for each individual in the data.

Data suitable for use with this package consists of columns of data
obtained from a set of individuals (e.g.~plants, pots, carts, plots or
units) over time. There should be a unique identifier for each
individual and a time variable, such as Days after Planting (DAP), that
contain no repeats for an individual. The combination of the identifier
and a time for an individual should be unique to that individual. For
imaging data, the individuals may be arranged in a grid of Lanes
\(\times\) Positions. That is, the minimum set of columns is an
individuals, a times and one or more primary trait columns.

The full set of functions falls into the following natural groupings:

\begin{enumerate}
\def\labelenumi{(\roman{enumi})}
\item
  Wrapper functions
\item
  Data
\item
  Plots
\item
  Smoothing and calculation of growth rates and water use traits for
  each individual
\item
  Data frame manipulation
\item
  General calculations
\item
  Principal variates analysis (PVA)
\end{enumerate}

\hypertarget{what-it-needs}{%
\subsection{What it needs}\label{what-it-needs}}

It imports \href{https://CRAN.R-project.org/package=dae}{dae},
\href{https://CRAN.R-project.org/package=GGally}{GGally},
\href{https://CRAN.R-project.org/package=ggplot2}{ggplot2},
\texttt{grDevices},
\href{https://CRAN.R-project.org/package=Hmisc}{Hmisc},
\href{https://CRAN.R-project.org/package=JOPS}{JOPS}, \texttt{methods},
\href{https://CRAN.R-project.org/package=RColorBrewer}{RColorBrewer},
\href{https://CRAN.R-project.org/package=readxl}{readxl},
\href{https://CRAN.R-project.org/package=reshape}{reshape},
\texttt{stats},
\href{https://CRAN.R-project.org/package=stringi}{stringi},
\texttt{utils}.

\hypertarget{license}{%
\subsection{License}\label{license}}

The \texttt{growthPheno} package is distributed under the
\href{https://opensource.org/licenses/GPL-2.0}{GPL (\textgreater= 2)
licence}.

\end{document}
